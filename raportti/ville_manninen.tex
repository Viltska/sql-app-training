\documentclass[12pt,a4paper]{article}
\usepackage[utf8]{inputenc}
\usepackage{lmodern}
\usepackage[T1]{fontenc}
\usepackage[finnish]{babel}

\begin{document}
\title{Tietokantojen perusteet, kevät 2020 \\
Harjoitustyön raportti}
\author{Ville Manninen \\014922187 \\Macville}
\date{\today}
\cleapage\maketitle
\thispagestyle{empty}


\tableofcontents
\newpage
\section{Esittely}
Harjoitustyö on toteutettu käyttämällä NetBeans ympäristöä hyödyntäen SQlite tietokantaa sekä Java Database Connectivity (JDBC) ohjelmointirajapintaa. Harjoitustyöllä on mahdollista suorittaa seuraavat ominaisuudet.
\begin{itemize}
\item Tietokannan taulukoiden luominen.
\item Uusien asiakkaiden, paikkojen ja pakettien lisääminen.
\end{itemize}



\newpage
\section{Tietokantakaavio ja SQL-skeema}

\newpage
\section{Tehokkuustesti}
\subsection*{Ilman indeksejä}
\subsection*{Indeksien lisäämisen jälkeen}

\newpage
\section{Tietokannan eheys}



\newpage
\section{Lähdekoodi}
Lähdekoodin GitHub repositorio \left \textbf{https://github.com/Viltska/sql-app-training} \right


\newpage

\end{document}